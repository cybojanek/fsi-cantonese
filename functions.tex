
%%%%%%%%%%%%%%%%%%%%%%%%%%%%%%%%%%%%%%%%%%%%%%%%%%%%%%%%%%%%%%%%%%%%%%%%%%%%%%%%

% Jyutping formatting which takes as input a string like xxxxNxxxxN and produces
% a string where the N are superscripted.
% \jping{ji4gaa1}
%
% @param string of the format xxxxNxxxxN
\newcounter{jpingStart}
\newcounter{jpingEnd}
\newcounter{jpingIndex}
\newcounter{jpingLength}
\newcounter{jpingLengthPlusOne}
\newcommand{\jping}[1]{%
    % NOTE: for single use, this would work, but we need to split...
    % \StrGobbleRight{#1}{1}\textsuperscript{\StrRight{#1}{1}}%
    % Get length of text
    \StrLen{#1}[\jpingLength]%
    \setcounter{jpingLengthPlusOne}{\jpingLength + 1}%
    % Set start of text
    \setcounter{jpingStart}{1}%
    \setcounter{jpingEnd}{0}%
    % Loop over each char
    \forloop{jpingIndex}{1}{\value{jpingIndex} < \arabic{jpingLengthPlusOne}}{%
        % Get char
        \StrChar{#1}{\arabic{jpingIndex}}[\jpingChar]%
        % If numeric, then handle
        % NOTE: include 7 for high falling examples
        \IfSubStr{1234567*}{\jpingChar}{%
            \text{\textsuperscript{\jpingChar}}%
        }{%
            \text{\jpingChar}%
        }%
    }%
}%

%%%%%%%%%%%%%%%%%%%%%%%%%%%%%%%%%%%%%%%%%%%%%%%%%%%%%%%%%%%%%%%%%%%%%%%%%%%%%%%%

% Variables for dtext
\newcounter{dtextStart}
\newcounter{dtextEnd}
\newcounter{dtextIndex}
\newcounter{dtextSubWordPlusOne}
\newcounter{dtextLength}
\newcounter{dtextLengthPlusOne}
\newcounter{dtextSubWord}
\newcounter{dtextNormal}

% Double text, meaning cantonese characters on top, and jyutping ont he bottom.
% The characters and jyutping will be grouped by spaces, with punctuation
% (including chinese punctuation), correctly ignored and skipped. Uses \jping
% for each jyutping group.
%
% Test cases
% \dtext{而家 我 問 你,你 答 我。}{ji4gaa1 ngo5 man6 nei5 nei5 daap3 ngo5}\\
% \dtext{而家 我 問 你,你 答 我。X}{ji4gaa1 ngo5 man6 nei5 nei5 daap3 ngo5 w}\\
% \dtext{而家}{ji4gaa1}\\
% \dtext{而家 我 問 你,你 答 我。a}{ji4gaa1 ngo5 man6 nei5 nei5 daap3 ngo5 -}\\
% \dtext{而家 我 問 你,你 答 我。X b}{ji4gaa1 ngo5 man6 nei5 nei5 daap3 ngo5 w -}\\
% \dtext{而家}{-}\\
% \dtext{而家。。。}{ji4gaa1}\\
% \dtext{而家 我 問 你,,你 答 我。。。}{ji4gaa1 ngo5 man6 nei5 nei5 daap3 ngo5}\\
% TODO: Mode: top only, top and bottom, bottom only, top and bottom white
%
% @param cantonese characters
% @param jyutping
\newcommand{\dtext}[2]{%
    % Get length of text
    \StrLen{#1}[\dtextLength]%
    \setcounter{dtextSubWord}{0}%
    \setcounter{dtextNormal}{0}%
    \setcounter{dtextLengthPlusOne}{\dtextLength + 1}%
    % Count number of spaces
    \StrCount{#2}{ }[\dtextSpaces]
    % Set start of word
    \setcounter{dtextStart}{1}%
    \setcounter{dtextEnd}{0}%
    % Loop over each char
    \forloop{dtextIndex}{1}{\value{dtextIndex} < \arabic{dtextLengthPlusOne}}{%
        % Get char
        \StrChar{#1}{\arabic{dtextIndex}}[\dtextChar]%
        % If special, then handle
        % https://en.wikipedia.org/wiki/Chinese_punctuation
        \IfSubStr{; 。「」﹁﹂"、‧《》〈〉﹏…——~,?;,!-}{\dtextChar}{%
            % End of word
            \StrMid{#1}{\arabic{dtextStart}}{\arabic{dtextEnd}}[\textOver]%
            \stepcounter{dtextEnd}%
            \setcounter{dtextStart}{\arabic{dtextEnd}}%
            \stepcounter{dtextStart}%
            \ifthenelse{\equal{\arabic{dtextNormal}}{\string 0}}{%
                % No characters! Just print this one and continue
            }{%
                % Add subtext
                \ifthenelse{\equal{\arabic{dtextSubWord}}{\string 0}}{%
                    % Handle first case
                    \StrCut[1]{#2}{ }{\textUnder}{\rightpart}%
                }{%
                    % Handle last case
                    \ifthenelse{\equal{\arabic{dtextSubWord}}{\dtextSpaces}}{%
                        % IF ALL LAST .... then wont be caught...
                        \StrCut[\arabic{dtextSubWord}]{#2}{ }{\leftpart}{\textUnder}%
                    }{% Handle middle case
                        \setcounter{dtextSubWordPlusOne}{\arabic{dtextSubWord} + 1}%
                        \StrBetween[\arabic{dtextSubWord},\arabic{dtextSubWordPlusOne}]{#2}{ }{ }[\textUnder]%
                    }%
                }%
                % Handle ignore
                \ifthenelse{\equal{\textUnder}{\string -}}{%
                    \text{\textOver}%
                }{%
                    $\underset{\jping{\textUnder}}{\text{\textOver}}$%
                }%
                % \text{[\textUnder][\textOver][\dtextChar]}
                \stepcounter{dtextSubWord}%
                \setcounter{dtextNormal}{0}%
            }%
            % Enable wrapping with stupid spaces...
            \ifthenelse{\equal{\dtextChar}{ }}{%
                \text{} %
            }{%
                \text{\dtextChar}%
            }%
        }{%
            % Not end of word
            \stepcounter{dtextEnd}%
            \stepcounter{dtextNormal}%
            % Handle last not being special
            \ifthenelse{\equal{\arabic{dtextIndex}}{\dtextLength}}{%
                % End of word
                \StrMid{#1}{\arabic{dtextStart}}{\arabic{dtextEnd}}[\textOver]%
                \stepcounter{dtextEnd}%
                \setcounter{dtextStart}{\arabic{dtextEnd}}%
                \stepcounter{dtextStart}%
                % Special case of single word
                \ifthenelse{\equal{\arabic{dtextSubWord}}{\string 0}}{%
                    % FIXME: lazy...
                    \StrMid{#2}{0}{9999}[\textUnder]%
                }{%
                    \StrCut[\arabic{dtextSubWord}]{#2}{ }{\leftpart}{\textUnder}%
                }%
                % Handle ignore
                \ifthenelse{\equal{\textUnder}{\string -}}{%
                    \text{\textOver}%
                }{%
                    $\underset{\jping{\textUnder}}{\text{\textOver}}$%
                }%
            }{%
            }%
        }%
    }%
}

%%%%%%%%%%%%%%%%%%%%%%%%%%%%%%%%%%%%%%%%%%%%%%%%%%%%%%%%%%%%%%%%%%%%%%%%%%%%%%%%

% Create two column, auto numbered classroom phrases, ie
% \classroomPhrases{
%   {而家踏幾呀}{ready yet}
%   {而家踏幾呀}{ready yet}
%   {\dtext{而家踏幾呀}{ji4 gaa1 daap3 gei2 aa3}}{ready yet}
% }
%
% @param array of tuples
\newcommand*{\classroomPhrases}[1]{%
    \setcounter{classroomPhraseCounter}{1}
    \relax
    \renewcommand{\arraystretch}{2}
    \begin{tabularx}{\linewidth}{l X l X}
    \classroomPhrase#1\relax\relax
    \end{tabularx}
    \renewcommand{\arraystretch}{1}
}
\newcounter{classroomPhraseCounter}
% Helper method for classroomPhrases
\newcommand{\classroomPhrase}[2]{%
    \ifx\relax#1\\\empty%
    \else%
    \noindent\arabic{classroomPhraseCounter}. & #1 & \arabic{classroomPhraseCounter}. & #2\\\relax%
    \stepcounter{classroomPhraseCounter}%
    \expandafter\classroomPhrase%
    \fi%
}

%%%%%%%%%%%%%%%%%%%%%%%%%%%%%%%%%%%%%%%%%%%%%%%%%%%%%%%%%%%%%%%%%%%%%%%%%%%%%%%%

% Create three column, auto numbered vocabulary list, ie
% \vocabularyChecklist{
%   {A}{ex}{oh}
%   {Chahn}{sur}{Chan}
%   {deimyjhyu}{ph}{Excuse me; I beg your pardon; I'm sorry}
%   {而家踏幾呀}{abc}{ready yet}
% }
%
% @param array of triples
\newcommand*{\vocabularyChecklist}[1]{%
    \setcounter{vocabularyChecklistCounter}{1}
    \relax
    \renewcommand{\arraystretch}{2}
    \begin{tabularx}{\linewidth}{r l r X}
    \vocabularyEntry#1\relax\relax\relax
    \end{tabularx}
    \renewcommand{\arraystretch}{1}
}
\newcounter{vocabularyChecklistCounter}
% Helper method for vocabularyChecklist
\newcommand{\vocabularyEntry}[3]{%
    \ifx\relax#1\\\empty%
    \else%
    \arabic{vocabularyChecklistCounter}. & #1 & #2: & #3\\\relax%
    \stepcounter{vocabularyChecklistCounter}%
    \expandafter\vocabularyEntry%
    \fi%
}

%%%%%%%%%%%%%%%%%%%%%%%%%%%%%%%%%%%%%%%%%%%%%%%%%%%%%%%%%%%%%%%%%%%%%%%%%%%%%%%%

% Create a drill
% NOTE: does not use simple nested stuff, due to ifthenelse and multicolumn
% errors on omit

% \drillExample{
%    \drillExampleEntry {T} {\dtext{李 太,早晨}{lei5 taai2 zou2san4}} {Good morning, Mrs. Lee.}
%    \drillExampleEntry {S} {\dtext{李 太,再見}{lei5 taai2 zoi3gin3}} {Goodbye, Mrs. Lee.}
% }
% \drillExample{
%    \drillExampleEntrySub {T} {\dtext{李 太,早晨}{lei5 taai2 zou2san4}} {Good morning, Mrs. Lee.} {\dtext{陳}{can4}}
%    \drillExampleEntry {S} {\dtext{陳 太,再見}{can4 taai2 zoi3gin3}} {Good morning, Mrs. Chan.}
% }
\newcommand*{\drillExample}[1]{%
    \begin{minipage}{\linewidth}
        Ex:
        \relax
        \renewcommand{\arraystretch}{2}
        \begin{tabularx}{\linewidth}{l X | l X}
        #1\relax\relax
        \end{tabularx}
        \renewcommand{\arraystretch}{1}
    \end{minipage}
}

% Example drill entry
%
% @param speaker
% @param left
% @param right
\newcommand{\drillExampleEntry}[3]{%
    #1: & #2 & #1: & #3 \\%
}%

% Example drill entry translation
%
% @param left
% @param right
\newcommand{\drillExampleEntryTrans}[2]{%
    & #1 & & #2 \\%
}%

% Example drill entry with substitution
%
% @param speaker
% @param left
% @param right
% @param substitution
\newcommand{\drillExampleEntrySub}[4]{%
    #1. & #2 & #1. & #3 \\%
    & / #4 / & & \\%
}%

% \drill{
%    \drillEntrySub {1} {\dtext{陳 太,早晨}{can4 taai2 zou2san4}} {\dtext{李 太,早晨}{lei5 taai2 zou2san4}} {\dtext{李}{lei5}}
%    \drillEntry {2} {\dtext{劉 生,早晨}{lau4 saang1 zou2san4}}{\dtext{劉 生,再見} {lau4 saang1 zoi3gin3}}
%    \drillEntryTrans {Good morning, Mr. Lau.} {}
% }
\newcommand*{\drill}[1]{%
    \renewcommand{\arraystretch}{2}
    \begin{tabularx}{\linewidth}{l l | l l}
    #1\relax\relax
    \end{tabularx}
    \renewcommand{\arraystretch}{1}
}

% Drill entry
%
% @param number
% @param left
% @param right
\newcommand{\drillEntry}[3]{%
    #1. & #2 & #1. & #3 \\%
}%

% Drill entry translation
%
% @param left
% @param right
\newcommand{\drillEntryTrans}[2]{%
    & #1 & & #2 \\%
}%

% Drill entry with substitution
%
% @param number
% @param left
% @param right
\newcommand{\drillEntrySub}[4]{%
    #1. & #2 & #1. & #3 \\%
    & / #4 / & & \\%
}%

\newcommand*{\convDrill}[1]{%
    \renewcommand{\arraystretch}{2}
    \begin{tabularx}{\linewidth}{l l X | l l X}
    #1\relax\relax
    \end{tabularx}
    \renewcommand{\arraystretch}{1}
}

\newcommand{\convDrillEntry}[4]{%
    \IfSubStr{1234567890}{#1}{#1.}{} & #2: & #3 & \IfSubStr{1234567890}{#1}{#1.}{} & #2: & #4 \\%
}%


%%%%%%%%%%%%%%%%%%%%%%%%%%%%%%%%%%%%%%%%%%%%%%%%%%%%%%%%%%%%%%%%%%%%%%%%%%%%%%%%

% Create an audio tag
%
% @param tape side A/B
% @param timestamp in 0:00 format
\newcommand{\audioTag}[2]{%
    #1 $\triangleright$ #2%
}%

\newcommand{\subsText}[1]{%
    /#1/%
}

%%%%%%%%%%%%%%%%%%%%%%%%%%%%%%%%%%%%%%%%%%%%%%%%%%%%%%%%%%%%%%%%%%%%%%%%%%%%%%%%

% Create a converstation
% NOTE: does not use simple nested stuff, due to ifthenelse and multicolumn
% errors on omit
%
% \conversation{
%   %
%   \convBackground{(At the beginning of class in the morning)}
%   %
%   \convExplanation{\dtext{學生}{hok6saang1}}{student}
%   %
%   \cspeaker{\dtext{學生}{hok6saang1}}
%   \cbline{\dtext{何}{ho4}}{Ho, surname}
%   \cbline{\dtext{生}{saang1}}{Mr.}
%   \cbline{\dtext{何 生}{ho4 saang1}}{Mr. Ho}
%   \cbline{\dtext{早晨}{zou2san4}}{good morning}
%   \cfline{\dtext{何 生 早晨}{ho4 saang1 zou2san4}}{Good morning, Mr. Ho.}
%   \convExplanation{\dtext{先生}{sin1saang1}}{teacher}
%   %
%   \cspeaker{\dtext{先生}{sin1saang1}}
%   \cbline{\dtext{李}{lei5}}{Lee, surname}
%   \cbline{\dtext{太}{taai2}}{Mrs.}
%   \cbline{\dtext{李 太}{lei5 taai2}}{Mrs. Lee}
%   \cfline{\dtext{李 太 早晨。}{lei5 taai2 zou2san4}}{Good morning, Mrs. Lee.}
%   %
%   \convBackground{(At the end of the day, the students are leaving class.)}
%   %
% }
% @param table rows
\newcommand*{\conversation}[1]{%
    \relax
    \renewcommand{\arraystretch}{2}
    \begin{tabularx}{\linewidth}{X | X}
    #1\relax\relax
    \end{tabularx}
    \renewcommand{\arraystretch}{1}
}

% A line for a speaker change
%
% @param speaker name
\newcommand{\cspeaker}[1]{%
    \multicolumn{2}{c}{\textbf{\underline{#1}}} \\%
}%

% A line for a double width row (background information)
%
% @param text
\newcommand{\convBackground}[1]{%
    \multicolumn{2}{c}{#1} \\%
}%

% Buildup line
%
% @param left right
\newcommand{\cbline}[2]{%
    \indent #1 & \indent #2\\%
}%

% Full conversation line after a buildup
%
% @param left right
\newcommand{\cfline}[2]{%
    #1 & #2\\%
}%

% Additional conversation line (i.e. word explanation)
%
% @param left right
\newcommand{\convExplanation}[2]{%
    \indent #1 & \indent #2\\%
}%

%%%%%%%%%%%%%%%%%%%%%%%%%%%%%%%%%%%%%%%%%%%%%%%%%%%%%%%%%%%%%%%%%%%%%%%%%%%%%%%%

% Translater note
%
% @param text
\newcommand{\tnote}[1]{%
    \underline{Translation Note}: #1%
}

% TODO note
%
% @param text
\newcommand{\todo}[1]{%
    \textbf{\underline{TODO}}: #1%
}