\subsection{Basic Conversation}

\audioTag{A}{x:xx}

%%%%%%%%%%%%%%%%%%%%%%%%%%%%%%%%%%%%%%%%
\subsubsection{Buildup}

\conversation{
    %
    \convBackground{(At a party in Hong Kong)}
    %
    \convExplanation{\dtext{先生}{sin1saang1}}{man}
    %
    \cspeaker{\dtext{先生}{sin1saang1}}
    \cbline{\dtext{貴姓}{gwai3sing3}}{your surname (polite)}
    \cbline{\dtext{呀}{aa3}}{sentence suffix, to soften abruptness}
    \cbline{\dtext{小姐}{siu2ze2}}{woman}
    \cfline{\dtext{小姐 貴姓 呀?}{siu2ze2 gwai3sing3 aa3}}{What is your surname, Miss?}
    %
    \cspeaker{\dtext{小姐}{siu2ze2}}
    \cfline{\dtext{我 姓 王。}{ngo5 sing3 wong4}}{My name is Wong.}
    %
    \cspeaker{\dtext{先生}{sin1saang1}}
    \convBackground{(bowing slightly)}
    \cfline{\dtext{王 小姐。}{wong4 siu2ze2}}{Miss Wong.}
    %
    \cspeaker{\dtext{小姐}{siu2ze2}}
    \cbline{\dtext{呢}{ne1}}{sentence suffix for questions}
    \cfline{\dtext{先生 呢?}{sin1saang1 ne1}}{And you? (polite)}
    %
    \cspeaker{\dtext{先生}{sin1saang1}}
    \cbline{\dtext{小姓}{siu2sing3}}{my name (polite)}
    \cfline{\dtext{小姓 劉。}{siu2sing3 lau4}}{My name is Lau.}
    %
    \cspeaker{\dtext{小姐}{siu2ze2}}
    \convBackground{(bowing slightly)}
    \cfline{\dtext{劉 生。}{lau4 saang1}}{Mr. Lau.}
    %
    \cspeaker{\dtext{先生}{sin1saang1}}
    \convBackground{(Indicating a young lady standing beside Miss Wong)}
    \cbline{\dtext{乜嘢}{mat1je2} or \dtext{乜嘢}{me1je2} or \dtext{乜嘢}{mi1je2}}{what?}
    \cbline{\dtext{姓 乜嘢 呀?}{sing3 me1je2 aa3}}{have what surname?}
    \cbline{\dtext{朋友}{pang4jau5}}{friend}
    \cbline{\dtext{你}{nei5}}{your}
    \cbline{\dtext{你 朋友}{nei5 pang4jau5}}{your friend}
    \cfline{\dtext{你 朋友 姓 乜嘢 呀?}{nei5 pang4jau5 me1je2 aa3}}{What is your friend's name?}
    %
    \cspeaker{\dtext{小姐}{siu2ze2}}
    % NOTE: original book has maa4 which doesn't look right...
    \cbline{\dtext{姓 馬}{sing3 maa5}}{has the name Ma}
    % FIXME: indent
    \cbline{\dtext{嘅}{ge3}}{noun-forming boundword. ge suffixed to a Verb Phrase makes it grammatically a Noun Phrase.}
    \cbline{\dtext{姓 馬 嘅}{sing3 maa5 ge3}}{is a named-Ma one}
    \cfline{\dtext{佢 姓 馬 嘅。}{keoi5 sing3 maa5 ge3}}{Her name is Ma.}
    %
    \cspeaker{\dtext{先生}{sin1saang1}}
    \cbline{\dtext{廣東}{gwong2dung1}}{Kwangtung}
    \cbline{\dtext{人}{jan4}}{person}
    \cbline{\dtext{廣東人}{gwong2dung1jan4}}{Cantonese person, a person from Kwangtung province}
    \cbline{\dtext{係 唔係 呀?}{hai6 m4hai6 aa3}}{is/notis? a question formula}
    % FIXME: indent
    \cfline{\dtext{佢 係 唔係 廣東人 呀?}{keoi5 hai6 m4hai6 gwong2dung1jan4 aa3}}{Is she Cantonese?}
    %
    \cspeaker{\dtext{小姐}{siu2ze2}}
    \cbline{\dtext{上海}{soeng6hoi2}}{Shanghai}
    \cbline{\dtext{上海人}{soeng6hoi2jan4}}{Shanghai person}
    % FIXME: indent
    \cfline{\dtext{唔係 呀。佢 係 上海人。}{m4hai6 aa3 keoi5 hai6 soeng6hoi2jan4}}{No, she's from Shanghai.}
    %
    \cspeaker{\dtext{先生}{sin1saang1}}
    % FIXME: indent
    \cbline{\dtext{敢}{gam2}}{'Well then, ...', 'Say, ...', sentence prefix, resuming the thread from the previous discussion.}
    \cfline{\dtext{敢,你 呢?}{gam2 nei5 ne1}}{And you?}
    %
    \cspeaker{\dtext{小姐}{siu2ze2}}
    \cbline{\dtext{都}{dou1}}{also}
    \cbline{\dtext{都 係 上海人}{dou1 hai6 soeng6hoi2jan4}}{also am Shanghai person}
    \cfline{\dtext{我 都 係 上海人。}{ngo5 dou1 hai6 soeng6hoi2jan4}}{I'm also from Shanghai.}
}

%%%%%%%%%%%%%%%%%%%%%%%%%%%%%%%%%%%%%%%%
\subsubsection{Recapitulation}

\audioTag{A}{x:xx}

\conversation{
    %
    \convBackground{(At a party in Hong Kong)}
    %
    \convExplanation{\dtext{先生}{sin1saang1}}{man}
    %
    \cspeaker{\dtext{先生}{sin1saang1}}
    \cfline{\dtext{小姐 貴姓 呀?}{siu2ze2 gwai3sing3 aa3}}{What is your surname, Miss?}
    %
    \cspeaker{\dtext{小姐}{siu2ze2}}
    \cfline{\dtext{我 姓 王。}{ngo5 sing3 wong4}}{My name is Wong.}
    %
    \cspeaker{\dtext{先生}{sin1saang1}}
    \convBackground{(bowing slightly)}
    \cfline{\dtext{王 小姐。}{wong4 siu2ze2}}{Miss Wong.}
    %
    \cspeaker{\dtext{小姐}{siu2ze2}}
    \cfline{\dtext{先生 呢?}{sin1saang1 ne1}}{And you? (polite)}
    %
    \cspeaker{\dtext{先生}{sin1saang1}}
    \cfline{\dtext{小姓 劉。}{siu2sing3 lau4}}{My name is Lau.}
    %
    \cspeaker{\dtext{小姐}{siu2ze2}}
    \convBackground{(bowing slightly)}
    \cfline{\dtext{劉 生。}{lau4 saang1}}{Mr. Lau.}
    %
    \cspeaker{\dtext{先生}{sin1saang1}}
    \convBackground{(Indicating a young lady standing beside Miss Wong)}
    \cfline{\dtext{你 朋友 姓 乜嘢 呀?}{nei5 pang4jau5 me1je2 aa3}}{What is your friend's name?}
    %
    \cspeaker{\dtext{小姐}{siu2ze2}}
    \cfline{\dtext{佢 姓 馬 嘅。}{keoi5 sing3 maa5 ge3}}{Her name is Ma.}
    %
    \cspeaker{\dtext{先生}{sin1saang1}}
    \cfline{\dtext{佢 係 唔係 廣東人 呀?}{keoi5 hai6 m4hai6 gwong2dung1jan4 aa3}}{Is she Cantonese?}
    %
    \cspeaker{\dtext{小姐}{siu2ze2}}
    \cfline{\dtext{唔係 呀。佢 係 上海人。}{m4hai6 aa3 keoi5 hai6 soeng6hoi2jan4}}{No, she's from Shanghai.}
    %
    \cspeaker{\dtext{先生}{sin1saang1}}
    \cfline{\dtext{敢,你 呢?}{gam2 nei5 ne1}}{And you?}
    %
    \cspeaker{\dtext{小姐}{siu2ze2}}
    \cfline{\dtext{我 都 係 上海人。}{ngo5 dou1 hai6 soeng6hoi2jan4}}{I'm also from Shanghai.}
}