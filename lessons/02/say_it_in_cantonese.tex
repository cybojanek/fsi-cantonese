lessons/02/problem_sounds.\TeX\subsection{Say it in Cantonese}

In this section you get directed practice in using some of the Cantonese you have learned, using the English sentences to prompt you. This is not to be thought of as a translation exercise - the English is just to get you going. Try to put the ideas into Cantonese, saying it the way the Cantonese would. Often there will be quite a few ways to say the same thing.

\noindent Ask the person sitting next to you .. And he answers...

\drill{
	\drillEntry {1} {What is your name?} {My name is ...}
	\drillEntry {2} {Are you an Englishman?} {No I'm an American.}
	\drillEntry {3} {Is your friend also an American?} {Yes, he is.}
	\drillEntry {4} {Is Miss Ho from Shanghai?} {No, she's from Toishan.}
	\drillEntry {5} {Is Mr. Lau a Toishan man?} {Yes, he is.}
	\drillEntry {6} {What is your friend's name?} {His name is Lee.}
	\drillEntry {7} {Are you Mr(s). Wong?} {I'm not Mr(s). Wong, my name is ...}
	\drillEntry {8} {Are you a student?} {No, I'm not a student, I'm a teacher.}
}

\noindent At a party.

\begin{enumerate}
	\item Mr. Wong asks Mr. Ho his name.
	\item Mr. Ho replies that his name is Ho, and asks Mr. Wong his name.
	\item Mr. Wong gives his name, and asks Mr. Ho if he is a Kwangtung man.
	\item Mr. Ho answers that he is. He asks Mr. Ho if he also is from Kwangtung.
	\item Mr. Wong says no, that he is a Shanghai man.
\end{enumerate}

\noindent A and B, two new students, wait for the teacher to come to class.

\begin{enumerate}
	\item A asks B what his name is.
	\item B replies and inquires A's name.
	\item A gives his name, and asks B if he is Japanese.
	\item B replies, and asks A if he is an Englishman.
	\item A replies, and asks B what C's name is.
	\item B replies, adding that C is Chinese.
\end{enumerate}