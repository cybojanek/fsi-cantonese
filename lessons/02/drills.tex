\subsection{Drills}

%%%%%%%%%%%%%%%%%%%%%%%%%%%%%%%%%%%%%%%%

\begin{minipage}{\linewidth}

\paragraph{1. Transformation Drill:} Make negative sentences following the pattern of the example. Student should point to himself in \dtext{我}{ngo5} sentences, to another student in \dtext{佢}{keoi5} and \dtext{你}{nei5} sentences.

\audioTag{B}{}

\drillExample{
    \drillExampleEntry {T} {\dtext{佢 係 上海人。}{keoi5 hai6 soeng6hoi2jan4}} {He (or she) is from Shanghai.}
    \drillExampleEntry {S} {\dtext{佢 唔係 上海人。}{keoi5 m4hai6 soeng6hoi2jan4}} {He (or she) is not from Shanghai.}
}

\drill{
    \drillEntry {1} {\dtext{佢 係 上海人。}{keoi5 hai6 soeng6hoi2jan4}} {\dtext{佢 唔係 上海人。}{keoi5 m4hai6 soeng6hoi2jan4}}
    \drillEntry {2} {\dtext{佢哋 係 廣東人。}{keoi5dei6 hai6 gwong2dung1jan4}} {\dtext{佢哋 唔係 廣東人。}{keoi5dei6 m4hai6 gwong2dung1jan4}}
    \drillEntry {3} {\dtext{我 係 中國人。}{ngo5 hai6 zung1gwok3jan4}} {\dtext{我 唔係 中國人。}{ngo5 m4hai6 zung1gwok3jan4}}
    \drillEntry {4} {\dtext{我哋 係 中國人。}{ngo5dei6 hai6 zung1gwok3jan4}} {\dtext{我哋 唔係 中國人。}{ngo5dei6 m4hai6 zung1gwok3jan4}}
    \drillEntry {5} {\dtext{你 係 英國人。}{nei5 hai6 jing3gwok3jan4}} {\dtext{你 唔係 英國人。}{nei5 m4hai6 jing3gwok3jan4}}
    \drillEntry {6} {\dtext{你 係 美國人。}{nei5 hai6 mei5gwok3jan4}} {\dtext{你 唔係 美國人。}{nei5 m4hai6 mei5gwok3jan4}}
    \drillEntry {7} {\dtext{你哋 係 美國人。}{nei5dei6 hai6 mei5gwok3jan4}} {\dtext{你哋 唔係 美國人。}{nei5dei6 m4hai6 mei5gwok3jan4}}
    \drillEntry {8} {\dtext{我 係 日本人。}{ngo5 hai6 jat6bun2jan4}} {\dtext{我 唔係 日本人。}{ngo5 m4hai6 jat6bun2jan4}}
    \drillEntry {9} {\dtext{我 係 臺山人。}{ngo5 hai6 toi4saan1jan4}} {\dtext{我 唔係 臺山人。}{ngo5 m4hai6 toi4saan1jan4}}
}

\end{minipage}

%%%%%%%%%%%%%%%%%%%%%%%%%%%%%%%%%%%%%%%%

\begin{minipage}{\linewidth}

\paragraph{2. Substitution Drill: } Substitute the cue word to make a new sentence, following the pattern of the example.

\audioTag{B}{}

\drillExample{
    \drillExampleEntrySub {T} {\dtext{佢哋 係 廣東人。}{keoi5dei6 hai6 gwong2dung1jan4}} {They are Cantonese.} {\dtext{上海人}{soeng6hoi2jan4}}
    \drillExampleEntry {S} {\dtext{佢哋 係 上海人。}{keoi5dei6 hai6 soeng6hoi2jan4}} {They are Shanghai people.}
}

\drill{
    \drillEntrySub {1} {\dtext{佢哋 係 上海人。}{keoi5dei6 hai6 soeng6hoi2jan4}} {\dtext{佢哋 係 美國人。}{keoi5dei6 hai6 mei5gwok3jan4}} {\dtext{美國人}{mei5gwok3jan4}}
    \drillEntrySub {2} {\dtext{佢哋 係 美國人。}{keoi5dei6 hai6 mei5gwok3jan4}} {\dtext{佢哋 係 英國人。}{keoi5dei6 hai6 jing3gwok3jan4}} {\dtext{英國人}{jing3gwok3jan4}}
    \drillEntrySub {3} {\dtext{佢哋 係 英國人。}{keoi5dei6 hai6 jing3gwok3jan4}} {\dtext{佢哋 係 日本人。}{keoi5dei6 hai6 jat6bun2jan4}} {\dtext{日本人}{jat6bun2jan4}}
    \drillEntrySub {4} {\dtext{佢哋 係 日本人。}{keoi5dei6 hai6 jat6bun2jan4}} {\dtext{佢哋 係 中國人。}{keoi5dei6 hai6 zung1gwok3jan4}} {\dtext{中國人}{zung1gwok3jan4}}
    \drillEntrySub {5} {\dtext{佢哋 係 中國人。}{keoi5dei6 hai6 zung1gwok3jan4}} {\dtext{佢哋 係 廣東人。}{keoi5dei6 hai6 gwong2dung1jan4}} {\dtext{廣東人}{gwong2dung1jan4}}
}

\end{minipage}

%%%%%%%%%%%%%%%%%%%%%%%%%%%%%%%%%%%%%%%%

\begin{minipage}{\linewidth}

\paragraph{3. Mixed Substitution Drill} Substitute the cue word in the appropriate position, following the pattern of the example.

\audioTag{B}{}

\drillExample{
    \drillExampleEntrySub {T} {\dtext{我 係 上海人。}{ngo5 hai6 soeng6hoi2jan4}} {I am from Shanghai.} {\dtext{你哋}{nei5dei6}}
    \drillExampleEntry {S} {\dtext{你哋 係 上海人。}{nei5dei6 hai6 soeng6hoi2jan4}} {You (plu.) are from Shanghai.}
    \drillExampleEntrySub {T} {\dtext{你哋 係 上海人。}{nei5dei6 hai6 soeng6hoi2jan4}} {You (plu.) are from Shanghai.} {\dtext{中國人}{zung1gwok3jan4}}
    \drillExampleEntry {S} {\dtext{你哋 係 中國人。}{nei5dei6 hai6 zung1gwok3jan4}} {You (plu.) are Chinese.}
}

\drill{
    \drillEntrySub {1} {\dtext{佢 係 英國人。}{keoi5 hai6 jing3gwok3jan4}} {\dtext{佢哋 係 英國人。}{keoi5dei6 hai6 jing3gwok3jan4}} {\dtext{佢爹}{keoi5dei6}}
    \drillEntrySub {2} {\dtext{佢哋 係 英國人。}{keoi5dei6 hai6 jing3gwok3jan4}} {\dtext{你哋 係 英國人。}{nei5dei6 hai6 jing3gwok3jan4}} {\dtext{你哋}{nei5dei6}}
    \drillEntrySub {3} {\dtext{你哋 係 英國人。}{nei5dei6 hai6 jing3gwok3jan4}} {\dtext{你哋 係 美國人。}{nei5dei6 hai6 mei5gwok3jan4}} {\dtext{美國人}{mei5gwok3jan4}}
    \drillEntrySub {4} {\dtext{你哋 係 美國人。}{nei5dei6 hai6 mei5gwok3jan4}} {\dtext{我 係 美國人。}{ngo5 hai6 mei5gwok3jan4}} {\dtext{我}{ngo5}}
    \drillEntrySub {5} {\dtext{我 係 學生。}{ngo5 hai6 hok6saang1}} {\dtext{我 係 先生。}{ngo5 hai6 sin1saang1}} {\dtext{先生}{sin1saang1}}
}

\end{minipage}

%%%%%%%%%%%%%%%%%%%%%%%%%%%%%%%%%%%%%%%%

\begin{minipage}{\linewidth}

\paragraph{4. Expansion Drill:} Expand the cue sentences as indicated in the example. Students should gesture to indicate pronouns.

\audioTag{B}{}

\drillExample{
    \drillExampleEntry {T} {\dtext{佢 唔係 李 太。}{keoi5 m4hai6 lei5 taai2}} {She is not Mrs. Lee.}
    \drillExampleEntry {S} {\dtext{佢 唔係 李 太,我 係。}{keoi5 m4hai6 lei5 taai2 ngo5 hai6}} {She is not Mrs. Lee, I am.}
}

\drill{
    \drillEntry {1} {\dtext{佢 唔係 張 生。}{keoi5 m4hai6 zoeng1 saang1}} {\dtext{佢 唔係 張 生,我 係。}{keoi5 m4hai6 zoeng1 saang1 ngo5 hai6}}
    \drillEntry {2} {\dtext{佢 唔係 陳 小姐。}{keoi5 m4hai6 can4 siu2ze2}} {\dtext{佢 唔係 陳 小姐,我 係。}{keoi5 m4hai6 can4 siu2ze2 ngo5 hai6}}
    \drillEntry {3} {\dtext{佢 唔係 何 生。}{keoi5 m4hai6 ho4 saang1}} {\dtext{佢 唔係 何 生,我 係。}{keoi5 m4hai6 ho4 saang1 ngo5 hai6}}
    \drillEntry {4} {\dtext{佢 唔係 李 太。}{keoi5 m4hai6 lei5 taai2}} {\dtext{佢 唔係 李 太,我 係。}{keoi5 m4hai6 lei5 taai2 ngo5 hai6}}
    \drillEntry {5} {\dtext{佢 唔係 陳 生。}{keoi5 m4hai6 can4 saang1}} {\dtext{佢 唔係 陳 生,我 係。}{keoi5 m4hai6 can4 saang1 ngo5 hai6}}
}

\end{minipage}

%%%%%%%%%%%%%%%%%%%%%%%%%%%%%%%%%%%%%%%%

\begin{minipage}{\linewidth}

\paragraph{5. Transformation Drill:} Respond according to the pattern of the example. Students gesture pronouns.

\audioTag{B}{}

\drillExample{
    \drillExampleEntry {T} {\dtext{我 係 美國人。}{ngo5 hai6 mei5gwok3jan4}} {I am an American.}
    \drillExampleEntry {S} {\dtext{你 係 唔係 美國人。}{nei5 hai6 m4hai6 mei5gwok3jan4 aa3}} {Are you an American?}
}

\drill{
    \drillEntry {1} {\dtext{我 係 廣東人。}{ngo5 hai6 gwong2dung1jan4}} {\dtext{我 係 唔係 廣東人 呀?}{ngo5 hai6 m4hai6 gwong2dung1jan4 aa3}}
    \drillEntry {2} {\dtext{我 係 王 生。}{ngo5 hai6 wong4 saang1}} {\dtext{我 係 唔係 王 生 呀?}{ngo5 hai6 m4hai6 wong4 saang1 aa3}}
    \drillEntry {3} {\dtext{佢 係 李 生。}{keoi5 hai6 lei5 saang1}} {\dtext{佢 係 唔係 李 生 呀?}{keoi5 hai6 m4hai6 lei5 saang1 aa3}}
    \drillEntry {4} {\dtext{我 係 美國人。}{ngo5 hai6 mei5gwok3jan4}} {\dtext{我 係 唔係 美國人 呀?}{ngo5 hai6 m4hai6 mei5gwok3jan4 aa3}}
    \drillEntry {5} {\dtext{我哋 係 日本人。}{ngo5dei6 hai6 jat6bun2jan4}} {\dtext{我哋 係 唔係 日本人 呀?}{ngo5dei6 hai6 m4hai6 jat6bun2jan4 aa3}}
    \drillEntry {6} {\dtext{佢 係 中國人。}{keoi5 hai6 zung1gwok3jan4}} {\dtext{佢 係 唔係 中國人 呀?}{keoi5 hai6 m4hai6 zung1gwok3jan4 aa3}}
}

Do the above sentences as an expansion drill, expanding \dtext{朋友}{pang4jau5} thus:
\drillExample{
    \drillExampleEntry {T} {\dtext{我 係 廣東人。}{ngo5 hai6 gwong2dung1jan4}} {I am Cantonese.}
    \drillExampleEntry {S} {\dtext{我 朋友 係 廣東人。}{ngo5 pang4jau5 hai6 gwong2dung1jan4}} {My friend is Cantonese.}
}

\end{minipage}

%%%%%%%%%%%%%%%%%%%%%%%%%%%%%%%%%%%%%%%%

\begin{minipage}{\linewidth}

\paragraph{6. Response Drill:} Respond according to the pattern of the example.

\audioTag{B}{}

\drillExample{
    \drillExampleEntrySub {T} {\dtext{張 小姐 係 唔係 美國人 呀?}{zoeng1 siu2ze2 hai6 m4hai6 mei5gwok3jan4}} {Is Miss Cheung an American?} {\dtext{英國人}{jing3gwok3jan4}}
    \drillExampleEntry {S} {\dtext{唔係。佢 係 英國人。}{m4hai6 keoi5 hai6 jing3gwok3jan4}} {No, she's English.}
}

\drill{
    \drillEntrySub {1} {\dtext{你 係 唔係 英國人 呀?}{nei5 hai6 m4hai6 jing3gwok3jan4 aa3}} {\dtext{唔係。我 係 美國人。}{m4hai6 ngo5 hai6 mei5gwok3jan4}} {\dtext{美國人}{mei5gwok3jan4}}
    \drillEntrySub {2} {\dtext{你 係 唔係 上海人 呀?}{nei5 hai6 m4hai6 soeng6hoi2jan4 aa3}} {\dtext{唔係。我 係 廣東人。}{m4hai6 ngo5 hai6 gwong2dung1jan4}} {\dtext{廣東人}{gwong2dung1jan4}}
    \drillEntrySub {3} {\dtext{張 生 係 唔係 上海人 呀?}{zoeng1 saang1 hai6 m4hai6 soeng6hoi2jan4 aa3}} {\dtext{唔係。佢 係 上海人。}{m4hai6 keoi5 hai6 soeng6hoi2jan4}} {\dtext{上海人}{soeng6hoi2jan4}}
    \drillEntrySub {4} {\dtext{馬 太 係 唔係 英國人 呀?}{maa5 taai2 hai6 m4hai6 jing3gwok3jan4 aa3}} {\dtext{唔係。佢 係 美國人。}{m4hai6 keoi5 hai6 mei5gwok3jan4}} {\dtext{美國人}{mei5gwok3jan4}}
    \drillEntrySub {5} {\dtext{佢 係 唔係 上海人 呀?}{keoi5 hai6 m4hai6 soeng6hoi2jan4 aa3}} {\dtext{唔係。佢 係 臺山人}{m4hai6 keoi5 hai6 toi4saan1jan4}} {\dtext{臺山人}{toi4saan1jan4}}
}
\end{minipage}

%%%%%%%%%%%%%%%%%%%%%%%%%%%%%%%%%%%%%%%%

\begin{minipage}{\linewidth}

\paragraph{7. Conversation Exercise:} Carry on the suggested Conversation following the pattern of the example.

\audioTag{B}{}

\drillExample{
    \drillExampleEntry {A} {\dtext{你 係 唔係 張 小姐 呀?}{nei5 hai6 m4hai6 zoeng1 siu2ze2 aa3}} {Are you Miss Cheung?}
    \drillExampleEntry {B} {\dtext{唔係。我 姓 陳。}{m4hai6 ngo5 sing3 can4}} {No, my name is Chan.}
}

\convDrill{
    \convDrillEntry {1} {A} {... \dtext{陳 生 呀?}{can4 saang1 aa3}} {\dtext{你 係 唔係 陳 生 呀?}{nei5 hai6 m4hai6 can4 saang1 aa3}}
    \convDrillEntry {} {B} {... \dtext{何。}{ho4}} {\dtext{唔係 呀。我 姓 何。}{m4hai6 aa3 ngo5 sing3 ho4}}
    %
    \convDrillEntry {2} {A} {... \dtext{李 小姐 呀?}{lei5 siu2ze2 aa3}} {\dtext{你 係 唔係 李 小姐 呀?}{nei5 hai6 m4hai6 lei5 siu2ze2 aa3}}
    \convDrillEntry {} {B} {... \dtext{張。}{zoeng1}} {\dtext{唔係 呀。我 姓 張。}{m4hai6 aa3 ngo5 sing3 zoeng1}}
    %
    \convDrillEntry {3} {A} {... \dtext{陳 太 呀?}{can4 taai2 aa3}} {\dtext{你 係 唔係 陳 太 呀?}{nei5 hai6 m4hai6 can4 taai2 aa3}}
    \convDrillEntry {} {B} {... \dtext{何。}{ho4}} {\dtext{唔係 呀。我 姓 何。}{m4hai6 aa3 ngo5 sing3 ho4}}
    %
    \convDrillEntry {4} {A} {... \dtext{李 生 呀?}{lei5 saang1 aa3}} {\dtext{你 係 唔係 李 生 呀?}{nei5 hai6 m4hai6 lei5 saang1 aa3}}
    \convDrillEntry {} {B} {... \dtext{張。}{zoeng1}} {\dtext{唔係 呀。我 姓 張。}{m4hai6 aa3 ngo5 sing3 zoeng1}}
    %
    \convDrillEntry {5} {A} {... \dtext{何 小姐 呀?}{ho4 siu2ze2 aa3}} {\dtext{你 係 唔係 何 小姐 呀?}{nei5 hai6 m4hai6 ho4 siu2ze2 aa3}}
    \convDrillEntry {} {B} {... \dtext{陳}{can4}} {\dtext{唔係 呀。我 姓 陳。}{m4hai6 aa3 ngo5 sing3 can4}}
}

Continue, with student A using a name at random and student B using his own name in response.

\end{minipage}

%%%%%%%%%%%%%%%%%%%%%%%%%%%%%%%%%%%%%%%%

\begin{minipage}{\linewidth}

\paragraph{8. Response Drill:} Respond according to the pattern of the example.

\audioTag{B}{}

\drillExample{
    \drillExampleEntrySub {T} {\dtext{佢 姓 王。}{keoi5 sing3 wong4}} {Her name is Wong.} {\dtext{張}{zoeng1}}
    \drillExampleEntry {S} {\dtext{佢 唔係 姓 王,姓 張。}{keoi5 m4hai6 sing3 wong4 sing3 zoeng1}} {Her name is not Wong, it's Cheung.}
}

\drill{
    \drillEntrySub {1} {\dtext{佢 姓 張。}{keoi5 sing3 zoeng1}} {\dtext{佢 唔係 姓 張,姓 何。}{keoi5 m4hai6 sing3 zoeng1, sing3 ho4}} {\dtext{何}{ho4}}
    \drillEntrySub {2} {\dtext{佢 姓 何。}{keoi5 sing3 ho4}} {\dtext{佢 唔係 姓 何,姓 陳。}{keoi5 m4hai6 sing3 ho4, sing3 can4}} {\dtext{陳}{can4}}
    \drillEntrySub {3} {\dtext{佢 姓 陳。}{keoi5 sing3 can4}} {\dtext{佢 唔係 姓 陳,姓 李。}{keoi5 m4hai6 sing3 can4, sing3 lei5}} {\dtext{李}{lei5}}
    \drillEntrySub {4} {\dtext{佢 姓 李。}{keoi5 sing3 lei5}} {\dtext{佢 唔係 姓 李,姓 劉。}{keoi5 m4hai6 sing3 lei5, sing3 lau4}} {\dtext{劉}{lau4}}
    \drillEntrySub {5} {\dtext{佢 姓 馬。}{keoi5 sing3 maa5}} {\dtext{佢 唔係 姓 馬,姓 王。}{keoi5 m4hai6 sing3 maa5, sing3 wong4}} {\dtext{王}{wong4}}
}

\end{minipage}

%%%%%%%%%%%%%%%%%%%%%%%%%%%%%%%%%%%%%%%%

\begin{minipage}{\linewidth}

\paragraph{9. Response Drill:} Respond according to the pattern of the example.

\audioTag{B}{}

\drillExample{
    \drillExampleEntrySub {T} {\dtext{佢 係 唔係 姓 陳 㗎?}{keoi5 hai6 m4hai6 sing3 can4 gaa3}} {Is her name Chan?} {\dtext{何}{ho4}}
    \drillExampleEntry {S} {\dtext{唔係。佢 姓 何 嘅。}{m4hai6 keoi5 sing3 ho4 ge3}} {No, her name is Ho.}
}

\drill{
    \drillEntrySub {1} {\dtext{佢 係 唔係 姓 李 㗎?}{keoi5 hai6 m4hai6 sing3 lei5 gaa3}} {\dtext{唔係。佢 姓 陳 嘅。}{m4hai6 keoi5 sing3 can4 ge3}} {\dtext{陳}{can4}}
    \drillEntrySub {2} {\dtext{佢 係 唔係 姓 馬 㗎?}{keoi5 hai6 m4hai6 sing3 maa5 gaa3}} {\dtext{唔係。佢 姓 何 嘅。}{m4hai6 keoi5 sing3 ho4 ge3}} {\dtext{何}{ho4}}
    \drillEntrySub {3} {\dtext{佢 係 唔係 姓 張 㗎?}{keoi5 hai6 m4hai6 sing3 zoeng1 gaa3}} {\dtext{唔係。佢 姓 李 嘅。}{m4hai6 keoi5 sing3 lei5 ge3}} {\dtext{李}{lei5}}
    \drillEntrySub {4} {\dtext{佢 係 唔係 姓 陳 㗎?}{keoi5 hai6 m4hai6 sing3 can4 gaa3}} {\dtext{唔係。佢 姓 馬 嘅。}{m4hai6 keoi5 sing3 maa5 ge3}} {\dtext{馬}{maa5}}
    \drillEntrySub {5} {\dtext{佢 係 唔係 姓 何 㗎?}{keoi5 hai6 m4hai6 sing3 ho4 gaa3}} {\dtext{唔係。佢 姓 張 嘅。}{m4hai6 keoi5 sing3 zoeng1 ge3}} {\dtext{張}{zoeng1}}
}

Comment: Sentence suffix \dtext{㗎}{gaa3} is a fusion of \dtext{嘅}{ge3} + \dtext{呀}{aa3} = \dtext{㗎}{gaa3}. In the choice-type question form, \dtext{姓}{sing3} is preceded by \dtext{係 唔係}{hai6 m4hai6} to make the question.

\end{minipage}

%%%%%%%%%%%%%%%%%%%%%%%%%%%%%%%%%%%%%%%%

\begin{minipage}{\linewidth}

\paragraph{10. Expansion Drill:} Expand the cue sentences as indicated in the example.

\audioTag{B}{}

\drillExample{
    \drillExampleEntrySub {T} {\dtext{你哋 係 美國人。}{nei5dei6 hai6 mei5gwok3jan4}} {You are Americans} {\dtext{你哋}{nei5dei6}}
    \drillExampleEntry {S} {\dtext{你哋 係 美國人;我哋 都 係 美國人。}{nei5dei6 hai6 mei5gwok3jan4 ngo5dei6 dou1 hai6 mei5gwok3jan4}} {You are Americans; we are also Americans.}
}

\drill{
    \drillEntrySub {1} {\dtext{佢哋 係 英國人。}{keoi5dei6 hai6 jing3gwok3jan4}} {\dtext{佢哋 係 英國人;我哋 都 係 英國人。}{keoi5dei6 hai6 jing3gwok3jan4 ngo5dei6 dou1 hai6 jing3gwok3jan4}} {\dtext{我哋}{ngo5dei6}}
    \drillEntrySub {2} {\dtext{我哋 係 上海人。}{ngo5dei6 hai6 soeng6hoi2jan4}} {\dtext{我哋 係 上海人;佢哋 都 係 上海人。}{ngo5dei6 hai6 soeng6hoi2jan4 keoi5dei6 dou1 hai6 soeng6hoi2jan4}} {\dtext{佢哋}{keoi5dei6}}
    \drillEntrySub {3} {\dtext{王 太 係 我 朋友。}{wong4 taai2 hai6 ngo5 pang4jau5}} {\dtext{王 太 係 我 朋友;佢 都 係 我 朋友。}{wong4 taai2 hai6 ngo5 pang4jau5 keoi5 dou1 hai6 ngo5 pang4jau5}} {\dtext{佢}{keoi5}}
    \drillEntrySub {4} {\dtext{佢哋 係 廣東人。}{keoi5dei6 hai6 gwong2dung1jan4}} {\dtext{佢哋 係 廣東人;你哋 都 係 廣東人。}{keoi5dei6 hai6 gwong2dung1jan4 nei5dei6 dou1 hai6 gwong2dung1jan4}} {\dtext{你哋}{nei5dei6}}
    \drillEntrySub {5} {\dtext{我哋 係 陳 嘅。}{ngo5dei6 hai6 can4 ge3}} {\dtext{我哋 係 陳 嘅;我哋 都 係 陳 嘅。}{ngo5dei6 hai6 can4 ge3 keoi5dei6 dou1 hai6 can4 ge3}} {\dtext{佢哋}{keoi5dei6}}
}

\end{minipage}

%%%%%%%%%%%%%%%%%%%%%%%%%%%%%%%%%%%%%%%%

\newpage
% \begin{minipage}{\linewidth}

\paragraph{11. Conversation Exercise:} Carry on the suggested Conversation following the pattern of the example.

\drillExample{
    \drillExampleEntry {A} {\dtext{小姐 貴姓 呀?}{siu2ze2 gwai3sing3 aa3}} {(To a woman) What is your name?}
    \drillExampleEntry {B} {\dtext{小姓 何。}{siu2sing3 ho4}} {My name is Ho.}
    \drillExampleEntry {A} {\dtext{何 小姐。}{ho4 siu2ze2}} {Miss Ho.}
    %
    \drillExampleEntry {A} {\dtext{先生 貴姓 呀?}{sin1saang1 gwai3sing3 aa3}} {(To a man) What is your name?}
    \drillExampleEntry {B} {\dtext{小姓 劉。}{siu2sing3 lau4}} {My name is Lau.}
    \drillExampleEntry {A} {\dtext{劉 生。}{lau4 saang1}} {Mr. Lau.}
}

\convDrill{
    \convDrillEntry {1} {A} {\dtext{先生}{sin1saang1} ...?} {\dtext{先生 貴姓 呀?}{sin1saang1 gwai3sing3 aa3}}
    \convDrillEntry {} {B} {\dtext{李}{lei5}} {\dtext{小姓 李。}{siu2sing3 李}}
    \convDrillEntry {} {A} {...} {\dtext{李 生。}{lei5 saang1}}
    %
    \convDrillEntry {2} {A} {\dtext{先生}{sin1saang1} ...?} {\dtext{先生 貴姓 呀?}{sin1saang1 gwai3sing3 aa3}}
    \convDrillEntry {} {B} {\dtext{陳}{can4}} {\dtext{小姓 陳。}{siu2sing3 陳}}
    \convDrillEntry {} {A} {...} {\dtext{陳 生。}{can4 saang1}}
    %
    \convDrillEntry {3} {A} {\dtext{先生}{sin1saang1} ...?} {\dtext{先生 貴姓 呀?}{sin1saang1 gwai3sing3 aa3}}
    \convDrillEntry {} {B} {\dtext{張}{zoeng1}} {\dtext{小姓 張。}{siu2sing3 張}}
    \convDrillEntry {} {A} {...} {\dtext{張 生。}{zoeng1 saang1}}
    %
    \convDrillEntry {4} {A} {\dtext{小姐}{siu2ze2} ...?} {\dtext{小姐 貴姓 呀?}{siu2ze2 gwai3sing3 aa3}}
    \convDrillEntry {} {B} {\dtext{王}{wong4}} {\dtext{小姓 王。}{siu2sing3 王}}
    \convDrillEntry {} {A} {...} {\dtext{王 生。}{wong4 saang1}}
    %
    \convDrillEntry {5} {A} {\dtext{小姐}{siu2ze2} ...?} {\dtext{小姐 貴姓 呀?}{siu2ze2 gwai3sing3 aa3}}
    \convDrillEntry {} {B} {\dtext{何}{ho4}} {\dtext{小姓 何。}{siu2sing3 何}}
    \convDrillEntry {} {A} {...} {\dtext{何 生。}{ho4 saang1}}
    %
}

% \end{minipage}

%%%%%%%%%%%%%%%%%%%%%%%%%%%%%%%%%%%%%%%%

\begin{minipage}{\linewidth}

\paragraph{12. Conversation Exercise:} Carry on the suggested Conversation following the pattern of the example.

\drillExample{
    \drillExampleEntry {A} {\dtext{你 朋友 姓 乜嘢 呀?}{nei5 pang4jau5 sing3 me1je5 aa3}} {What is your friend's name?}
    \drillExampleEntry {B} {\dtext{佢 姓 王 嘅。}{keoi5 sing3 wong4 ge3}} {His name is Wong.}
}

\convDrill{
    \convDrillEntry {1} {A} {...} {\dtext{你 朋友 姓 乜嘢 呀?}{nei5 pang4jau5 sing3 me1je5 aa3}}
    \convDrillEntry {} {B} {... \dtext{何}{ho4} ...} {\dtext{佢 姓 何 嘅。}{keoi5 sing3 ho4 ge3}}
    %
    \convDrillEntry {2} {A} {...} {\dtext{你 朋友 姓 乜嘢 呀?}{nei5 pang4jau5 sing3 me1je5 aa3}}
    \convDrillEntry {} {B} {... \dtext{劉}{lau4} ...} {\dtext{佢 姓 劉 嘅。}{keoi5 sing3 lau4 ge3}}
    %
    \convDrillEntry {3} {A} {...} {\dtext{你 朋友 姓 乜嘢 呀?}{nei5 pang4jau5 sing3 me1je5 aa3}}
    \convDrillEntry {} {B} {... \dtext{王}{wong4} ...} {\dtext{佢 姓 王 嘅。}{keoi5 sing3 wong4 ge3}}
    %
    \convDrillEntry {4} {A} {...} {\dtext{你 朋友 姓 乜嘢 呀?}{nei5 pang4jau5 sing3 me1je5 aa3}}
    \convDrillEntry {} {B} {... \dtext{張}{zoeng1} ...} {\dtext{佢 姓 張 嘅。}{keoi5 sing3 zoeng1 ge3}}
    %
    \convDrillEntry {5} {A} {...} {\dtext{你 朋友 姓 乜嘢 呀?}{nei5 pang4jau5 sing3 me1je5 aa3}}
    \convDrillEntry {} {B} {... \dtext{李}{lei5} ...} {\dtext{佢 姓 李 嘅。}{keoi5 sing3 lei5 ge3}}
    %
}

\end{minipage}

%%%%%%%%%%%%%%%%%%%%%%%%%%%%%%%%%%%%%%%%

\newpage
% \begin{minipage}{\linewidth}

\paragraph{13. Conversation Drill:} Carry on the suggested Conversation following the pattern of the example.

\drillExample{
    \drillExampleEntry {A} {\dtext{你 朋友 姓 乜嘢 呀?}{nei5 pang4jau5 sing3 me1je5 aa3}} {What is your friend's name?}
    \drillExampleEntry {B} {\dtext{佢 姓 王 嘅。}{keoi5 sing3 wong4 ge3}} {His name is Wong.}
    \drillExampleEntry {A} {\dtext{佢 係 唔係 廣東人 呀?}{keoi5 hai6 m4hai6 gwong2dung1jan4 aa3}} {Is he a Cantonese?}
    \drillExampleEntry {B} {\dtext{唔係。佢 係 日本人。}{m4hai6 keoi5 hai6 jat6bun2jan4}} {No, he's a Japanese.}
}

\convDrill{
    \convDrillEntry {1} {A} {...?} {\dtext{你 朋友 姓 乜嘢 呀?}{nei5 pang4jau5 sing3 me1je5 aa3}}
    \convDrillEntry {} {B} {... \dtext{何。}{ho4}} {\dtext{佢 姓 何 嘅。}{keoi5 sing3 ho4 ge3}}
    \convDrillEntry {} {A} {... \dtext{英國人 呀?}{jing3gwok3jan4 aa3}} {\dtext{佢 係 唔係 英國人 呀?}{keoi5 hai6 m4hai6 jing3gwok3jan4 aa3}}
    \convDrillEntry {} {B} {... \dtext{美國人。}{mei5gwok3jan4}} {\dtext{唔係。佢 係 美國人。}{m4hai6 keoi5 hai6 mei5gwok3jan4}}
    %
    \convDrillEntry {2} {A} {...?} {\dtext{你 朋友 姓 乜嘢 呀?}{nei5 pang4jau5 sing3 me1je5 aa3}}
    \convDrillEntry {} {B} {... \dtext{李。}{lei5}} {\dtext{佢 姓 李 嘅。}{keoi5 sing3 lei5 ge3}}
    \convDrillEntry {} {A} {... \dtext{上海人 呀?}{soeng6hoi2jan4 aa3}} {\dtext{佢 係 唔係 上海人 呀?}{keoi5 hai6 m4hai6 soeng6hoi2jan4 aa3}}
    \convDrillEntry {} {B} {... \dtext{臺山人。}{toi4saan1jan4}} {\dtext{唔係。佢 係 臺山人。}{m4hai6 keoi5 hai6 toi4saan1jan4}}
    %
    \convDrillEntry {3} {A} {...?} {\dtext{你 朋友 姓 乜嘢 呀?}{nei5 pang4jau5 sing3 me1je5 aa3}}
    \convDrillEntry {} {B} {... \dtext{陳。}{can4}} {\dtext{佢 姓 陳 嘅。}{keoi5 sing3 can4 ge3}}
    \convDrillEntry {} {A} {... \dtext{美國人 呀?}{mei5gwok3jan4 aa3}} {\dtext{佢 係 唔係 美國人 呀?}{keoi5 hai6 m4hai6 mei5gwok3jan4 aa3}}
    \convDrillEntry {} {B} {... \dtext{英國人。}{jing3gwok3jan4}} {\dtext{唔係。佢 係 英國人。}{m4hai6 keoi5 hai6 jing3gwok3jan4}}
    %
    \convDrillEntry {4} {A} {...?} {\dtext{你 朋友 姓 乜嘢 呀?}{nei5 pang4jau5 sing3 me1je5 aa3}}
    \convDrillEntry {} {B} {... \dtext{馬。}{maa5}} {\dtext{佢 姓 馬 嘅。}{keoi5 sing3 maa5 ge3}}
    \convDrillEntry {} {A} {... \dtext{廣東人 呀?}{gwong2dung1jan4 aa3}} {\dtext{佢 係 唔係 廣東人 呀?}{keoi5 hai6 m4hai6 gwong2dung1jan4 aa3}}
    \convDrillEntry {} {B} {... \dtext{上海人。}{soeng6hoi2jan4}} {\dtext{唔係。佢 係 上海人。}{m4hai6 keoi5 hai6 soeng6hoi2jan4}}
    %
    \convDrillEntry {5} {A} {...?} {\dtext{你 朋友 姓 乜嘢 呀?}{nei5 pang4jau5 sing3 me1je5 aa3}}
    \convDrillEntry {} {B} {... \dtext{王。}{wong4}} {\dtext{佢 姓 王 嘅。}{keoi5 sing3 wong4 ge3}}
    \convDrillEntry {} {A} {... \dtext{日本人 呀?}{jat6bun2jan4 aa3}} {\dtext{佢 係 唔係 日本人 呀?}{keoi5 hai6 m4hai6 jat6bun2jan4 aa3}}
    \convDrillEntry {} {B} {... \dtext{中國人。}{zung1gwok3jan4}} {\dtext{唔係。佢 係 中國人。}{m4hai6 keoi5 hai6 zung1gwok3jan4}}
    %
    \convDrillEntry {6} {A} {...?} {\dtext{你 朋友 姓 乜嘢 呀?}{nei5 pang4jau5 sing3 me1je5 aa3}}
    \convDrillEntry {} {B} {... \dtext{張。}{zoeng1}} {\dtext{佢 姓 張 嘅。}{keoi5 sing3 zoeng1 ge3}}
    \convDrillEntry {} {A} {... \dtext{上海人 呀?}{soeng6hoi2jan4 aa3}} {\dtext{佢 係 唔係 上海人 呀?}{keoi5 hai6 m4hai6 soeng6hoi2jan4 aa3}}
    \convDrillEntry {} {B} {... \dtext{日本人。}{jat6bun2jan4}} {\dtext{唔係。佢 係 日本人。}{m4hai6 keoi5 hai6 jat6bun2jan4}}
    %
}

% \end{minipage}

%%%%%%%%%%%%%%%%%%%%%%%%%%%%%%%%%%%%%%%%

\newpage
% \begin{minipage}{\linewidth}

\paragraph{14. Conversation Drill:} Carry on the suggested conversations following the pattern of the example.

\drillExample{
    \drillExampleEntry {A} {\dtext{先生 係 唔係 美國人 呀?}{sin1saang1 hai6 m4hai6 mei5gwok3jan4 aa3}} {Is the gentleman (i.e. Are you an American?)}
    \drillExampleEntry {B} {\dtext{唔係 - 我 係 英國人。小姐 呢?}{m4hai6 ngo5 hai6 jing3gwok3jan4 siu2ze2 ne1}} {No, I'm an Englishman. And the lady (i.e. you)?}
    \drillExampleEntry {A} {\dtext{我 係 廣東人。}{ngo5 hai6 gwong2dung1jan4}} {I am a Cantonese.}
}

\convDrill{
    \convDrillEntry {1} {A} {(woman): \dtext{先生}{sin1saang1} ... \dtext{廣東人。}{gwong2dung1jan4}} {\dtext{先生 係 唔係 廣東人 呀?}{sin1saang1 hai6 m4hai6 gwong2dung1jan4 aa3}}
    \convDrillEntry {} {B} {(man): ... \dtext{上海人。}{soeng6hoi2jan4}} {\dtext{唔係。我 係 上海人。小姐 呢?}{m4hai6 ngo5 hai6 soeng6hoi2jan4 siu2ze2 ne1}}
    \convDrillEntry {} {A} {(woman): ... \dtext{日本人。}{jat6bun2jan4}} {\dtext{我 係 日本人。}{ngo5 hai6 jat6bun2jan4}}
    %
    \convDrillEntry {2} {A} {(man): \dtext{小姐}{siu2ze2} ... \dtext{日本人。}{jat6bun2jan4}} {\dtext{小姐 係 唔係 日本人 呀?}{siu2ze2 hai6 m4hai6 jat6bun2jan4 aa3}}
    \convDrillEntry {} {B} {(woman): ... \dtext{中國人。}{zung1gwok3jan4}} {\dtext{唔係。我 係 中國人。小姐 呢?}{m4hai6 ngo5 hai6 zung1gwok3jan4 siu2ze2 ne1}}
    \convDrillEntry {} {A} {(man): ... \dtext{美國人。}{mei5gwok3jan4}} {\dtext{我 係 美國人。}{ngo5 hai6 mei5gwok3jan4}}
    %
    \convDrillEntry {3} {A} {(man): \dtext{先生}{sin1saang1} ... \dtext{英國人。}{jing3gwok3jan4}} {\dtext{先生 係 唔係 英國人 呀?}{sin1saang1 hai6 m4hai6 jing3gwok3jan4 aa3}}
    \convDrillEntry {} {B} {(man): ... \dtext{美國人。}{mei5gwok3jan4}} {\dtext{唔係。我 係 美國人。先生 呢?}{m4hai6 ngo5 hai6 mei5gwok3jan4 sin1saang1 ne1}}
    \convDrillEntry {} {A} {(man): ... \dtext{廣東人。}{gwong2dung1jan4}} {\dtext{我 係 廣東人。}{ngo5 hai6 gwong2dung1jan4}}
    %
    \convDrillEntry {4} {A} {(woman): \dtext{馬 先生}{maa5 sin1saang1} ... \dtext{日本人。}{jat6bun2jan4}} {\dtext{馬 先生 係 唔係 日本人 呀?}{maa5 sin1saang1 hai6 m4hai6 jat6bun2jan4 aa3}}
    \convDrillEntry {} {B} {(man): ... \dtext{中國人。}{zung1gwok3jan4}} {\dtext{唔係。我 係 中國人。小姐 呢?}{m4hai6 ngo5 hai6 zung1gwok3jan4 siu2ze2 ne1}}
    \convDrillEntry {} {A} {(woman): ... \dtext{英國人。}{jing3gwok3jan4}} {\dtext{我 係 英國人。}{ngo5 hai6 jing3gwok3jan4}}
    %
}

% \end{minipage}

