\section{Lesson 2}

\subsection{Classroom Phrases}

\paragraph{A.} Learn to respond to the following classroom instructions.

\audioTag{A}{0:07}

\classroomPhrases{
    {\dtext{而家 我 問 你,你 答 我}{ji4gaa1 ngo5 man6 nei5 nei5 daap3 ngo5}}
    {Now I'll ask you, and you answer me.}
    %
    {\dtext{而家 你哋 自己 問,自己 答}{ji4gaa1 nei5dei6 zi6gei2 man6 zi6gei2 daap6}}
    {Now you yourselve's ask and answer.}
    %
    {\dtext{繼續}{gai3zuk6}}
    {Continue. (i.e. Do the next one, keep going)}
    %
    {\dtext{你 做 A,你 做 B}{nei5 zou6 A nei5 zou6 B}}
    {You do A, you do B.}
}

\paragraph{B.} The following are some comments that the teacher may make on your recitations.

\audioTag{A}{0:50}

\classroomPhrases{
    {\dtext{啱 嘞}{ngaam1 laak3} OR \dtext{啱 嘞}{aam1 laak3}}
    {That's it. (After student succeeds in saying something right.)}
    %
    {\dtext{係 咁 嘞}{hai6 gam2 laak3}}
    {That's it. Now you've got it.}
    %
    {\dtext{係 嘞}{hai6 laak3}}
    {That's it. Now you've got it.}
    %
    {\dtext{好 準}{hou2 zeon2}}
    {Just right. Quite accurate.}
    %
    {\dtext{講得 好}{gong2dak1 hou2}}
    {Good, spoken well.}
    %
    {\dtext{講得 唔好}{gong2dak1 m4hou2}}
    {No that won't do. Not spoken right.}
    %
    {\dtext{差唔多}{caa1m4do1}}
    {Approximately. (i.e. Good enough for now, but not perfect.)}
    %
    {\dtext{要 熟 啲}{jiu3 suk6 di1}}
    {Get it smoother. (When a student's recitation is halting.)}
    %
    {\dtext{大聲 啲}{daai6seng1 di1}}
    {Louder.}
}
