\subsection{Notes}

\subsubsection{Culture Notes}

\begin{enumerate}
\item A \atext{廣東人} is a person from Kwangtung province. In English
such a person is usually referred to as 'Cantonese,' the English
name deriving from the city of Canton in Kwangtung province.
People from Hong Kong are also included in the term \atext{廣東人}.
\item Polite forms in social conversation
    \begin{itemize}
    \item \atext{先生} and \atext{小姐} are polite formal substitutes for \atext{你} 'you' as terms of direct address.
        \begin{enumerate}
            \item \atext{先生 貴姓 呀?} What is the gentleman's (i.e. your) name?
            \item \atext{小姐 貴姓 呀?} What is the lady's (i.e. your) name?
        \end{enumerate}
    \item \atext{小姐} is the general polite substitue for \atext{你} when addressing a woman, even if she is a married woman. Ex: Mr. Lee (to Mrs. Chan):\\
    \atext{小姐 係 唔係 廣東人 呀?}\\
    Is the lady (i.e. Are you from Kwangtung?)

    \item Surname and title as polite formal substitue for atext{你} as term of address. Ex: Mr Lee (to Miss Chan):\\
    \atext{陳 小姐 係 唔係 廣東人 呀?}\\
    Is Miss Chan (i.e., Are you from Kwangtung?)

    \item \atext{貴-} and \atext{小-}
        \begin{itemize}
            \item \atext{貴-} is a polite form meaning "your", referring to the person you are talking to.\\
            Ex: \atext{貴姓} = your name. The literal meaning of \atext{貴-} "precious, valuable".

            \item \atext{小-} is a polite form used in refering to oneself when talking with another preson. It means "my".\\
            Ex: \atext{小姓} = my name. The literal meaning of \atext{小-} is "small".

            \item \atext{我姓} seems more commonly used than \atext{小姓}, but \atext{貴姓} is more common than \atext{你 姓 乜嘢 呀?} in social conversation. At a doctor's office, or in registering for school, 'What is your name' would be more apt to be asked as \atext{姓 乜嘢?} than as \atext{貴姓 呀?}.
        \end{itemize}
    \end{itemize}
\end{enumerate}

\subsubsection{Structure Notes}
Some people in speaking about Cantonese and other Chinese languages, say "Cantonese has no grammar." In this they are referring to the fact that words in Cantonese (and other Chinese languages) do not undergo the changes of form which English words experience in relation to tense: see, saw, seen; to number: boy, boys; to case: I, me, my, mine; to word class: photograph, photographer, photography, photographic; to subject-verb concord: He sits, They sit.

\begin{enumerate}
    \item \underline{Verb form}: Absence of Subject-verb concord.\\
    There is no subject-verb concord in Cantonese. Whereas the English verb changes form in concord with the subject - I am, You are, He is - , the Cantonese verb remains in one form regardless of the subject.\\
    Ex:\\
    \renewcommand{\arraystretch}{2}
    \begin{tabular}{l l l l}
    Subject & Verb & & \\
    \atext{我} & \atext{係} & \atext{陳 小姐} & I am Miss Chan.\\
    \atext{我} & \atext{係} & \atext{我 朋友} & You are my friend.\\
    \atext{我} & \atext{係} & \atext{廣東人} & He is Cantonese.\\
    \atext{我} & \atext{係} & \atext{上海人} & They are Shanghai people.\\
    \end{tabular}
    \renewcommand{\arraystretch}{1}

    \item \underline{Noun form}: Absence of Singular/Plural Distinction.\\
    There is no distinction in Chinese nouns between singular and plural. One form is used for both single and plural objects, with other parts of the sentence, or sometimes simply the situational context, giving information regarding number.\\
    Ex:\\
    \renewcommand{\arraystretch}{2}
    \begin{tabular}{l l}
    \atext{佢 係 英國人。} & He is an Englishman. \\
    \atext{佢哋 係 英國人。} & They are Englishmen. \\
    \end{tabular}
    \renewcommand{\arraystretch}{1}

    \item \underline{Pronoun forms}.

    \begin{enumerate}
        \item Cantonese has three personal pronouns:
        \begin{enumerate}
            \item \atext{我} = I, me, my
            \item \atext{你} = you, your (singular)
            \item \atext{佢} = he, she, it, him, her
        \end{enumerate}
        \item Plurality is marked in personal pronouns by the plural suffix \atext{-哋}:
        \begin{enumerate}
            \item \atext{我哋} = we (both inclusive, and exclusive)
            \item \atext{你哋} = you (plural)
            \item \atext{佢哋} = they
        \end{enumerate}
    \end{enumerate}

    \item \underline{Modification structures: Noun modification}:

    In Cantonese a modifier precedes the noun it modifies:

    Example: modifier + noun head

    \atext{我 朋友 係 英國人。} My friend is an Englishman

    We will refer to this modifier-modified noun structure as a Noun Phrase (NP), consisting of modified and head.

    \item \underline{Sentence suffixes}.

    What we call sentence suffixes are also called "final particles" and "sentence finals".

    Sentence suffixes are used in conversation, and are a means by which the speaker signals the listener what he feels about what he's saying - that he is doubtful, definite, surprised, sympathetic, that he means to be polite, or sarcastic.

    Some sentence suffixes have actual content meaning. For example, \atext{咩}, which you will learn in Lesson 3, has interrogative meaning, and suffixed to a statement sentence makes it a question. But others operate primarily as described above - to add an emotion-carrying coloration to the sentences they attach to. As such they have been called also "intonation-carrying particles," intonation here used in its "tone of voice" sense.

    Two sentence suffixes appear in the Basic Conversation of this lesson:

    \begin{enumerate}
        \item Sentence suffix \atext{呀}

        The sentence suffix \atext{呀} has the effect of softening the sentence to which it is attached, making it less abrupt than it would other be.

        Examples from this lesson:

        \renewcommand{\arraystretch}{2}
        \begin{tabular}{l l l}
            1. & \atext{佢 係 唔係 廣東人 呀?} & Is she a Cantonese? \\
            2. & \atext{唔係 呀。} & No. \\
        \end{tabular}
        \renewcommand{\arraystretch}{1}

        In English a courteous tone of voice is perhaps the best counterpart to the \atext{呀} sentence suffix.

        \item Sentence suffix \atext{呢}

        \atext{呢} in a follow sentence sentence of structure "noun + \atext{呢?}" is an interrogative sentence suffix, meaning 'how about...?', 'And...?'. In such a sentence \atext{呢} is interrogative on its own:

        Example:

        \atext{我 係 廣東人;你 呢?} I am a Cantonese; how about you?

        Sentence suffix \atext{呀} is not substitutable for \atext{呢} in this sentence, \atext{呀} not having an interrogative sense of its own.

        We have used tone marks in writing the sentence suffixes, but perhaps it would have been better to use other symbols, maybe arrows pointing up for high, diagonally for rising, to the right for mid, down for falling. Since some finals can be said with different pitch contours with the effect of changing the coloration of what is said but not the content, they are not truly tonal words. For example, sentence suffix 呀, encountered in this lesson, we have described as having the effect of softening an otherwise rather abrupt sentence. This final can also be said at high pitch: \dtext{呀}{aa1}, without changing the sentence-softening aspect, but adding liveliness to the response.

        Example:

        \renewcommand{\arraystretch}{2}
        \begin{tabular}{l l l}
            1. & \atext{你 係 唔係 廣東人 呀?} & Are you a Cantonese? \\
            2. & \dtext{唔係 呀。}{m4hai6 aa1}\atext{我 係 上海人。} & No, siree, I'm a Shanghai man. \\
        \end{tabular}
        \renewcommand{\arraystretch}{1}

        Beginning students, even advanced students, often have a lot of difficulty with sentence suffixes, because they don't fit into categories which we recognize in English. Partly this is because most of us haven' t analyzed the English we use. How would you explain, for example, the English "sentences suffixes" in the following:

        \begin{enumerate}
            \item What do you mean by that, pray?
            \item Hand me that pencil, will you?
            \item Cut that out, hear?
            \item He's not coming, I don't think.
        \end{enumerate}

        Our advice to students in regard to sentence suffixes is absorb them as you can, don't get bogged down in trying to plumb their "real" meanings - in doing so, you spend more time on them than they warrant.

        \item \underline{Choice-type Questions}.

        Questions which in English would be answered by yes or no, are formed in Cantonese by coupling the positive and negative forms of a verb together, and requiring an echo answer of the suitable one. This question form we call the Choice-type Question.

        Example:

        \renewcommand{\arraystretch}{2}
        \begin{tabular}{l l l}
        Question: & \atext{佢 係 唔係 美國人 呀?} & Is he an American? \\
        Responses: & \atext{係} & Yes \\
        & \atext{唔係} & No. \\
        \end{tabular}
        \renewcommand{\arraystretch}{1}

        \item \underline{Question-word Questions}.

        Question-word Questions are question sentences using the Cantonese question-word equivelants of what, when, where, why, how, how much, how many, who. \atext{乜嘢?} (variant pronounciations \dtext{乜嘢?}{mat1je5} and \dtext{乜野?}{mi1je5}) is an example of a question word.

        In Cantonese question-word (QW) questions pattern like statement sentences - they have the same word order as statement sentences, with the question-word occupying the same position in the sentence which the reply word occupies in the statement.

        Example:

        \renewcommand{\arraystretch}{2}
        \begin{tabular}{l l}
        \atext{佢 姓 乜嘢 呀?} & [He is surnamed what?] What is his name? \\
        \atext{佢 姓 王。} & [He is surnamed Wong.] His name is Wong.
        \end{tabular}
        \renewcommand{\arraystretch}{1}

        \item \atext{-嘅}, noun-forming boundword

        \atext{嘅} attaches to the end of a word or phrase which is not a noun and makes it into a noun phrase. In such cases it usually works to translate \atext{-嘅} into English as 'one who' or 'such a one.' When we say \atext{嘅} is a boundword we mean it is never spoken as a one-word sentence, but always accompanies some other word.

        Example:

        \atext{佢 姓 王 嘅。}
        She is one who has the surname Wong. \underline{or} She's a person named Wong.

        \atext{㗎} is a fuesgion of \atext{嘅} + sentence suffix \atext{呀}.

        Example:

        \renewcommand{\arraystretch}{2}
        \begin{tabular}{l l}
        A: \atext{佢 係 唔係 姓 王 㗎?} & Is he named Wong? \\
        B: \atext{唔係 - 佢 唔係 姓 王 嘅。佢 姓 何。} & No, he's not named Wong. His name is Ho.
        \end{tabular}
        \renewcommand{\arraystretch}{1}

        \item \atext{乜嘢}, \dtext{乜嘢}{mat1je5}, \dtext{乜野}{mi1je5} = variant pronounciations for 'what?'

        \dtext{乜嘢}{mat1je5} is ocassionally used in conversations as an emphatic form; normally the spoken pronounciation is \atext{乜嘢} or \dtext{乜野}{mi1je5}, some people favoring \atext{乜嘢}, others \dtext{乜野}{mi1je5}. We have written \atext{乜嘢} uniformally in the text, but on the tapes you will hear all three forms.


    \end{enumerate}

\end{enumerate}
