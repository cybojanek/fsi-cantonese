\subsection{Basic Conversation}

\audioTag{A}{1:43}

%%%%%%%%%%%%%%%%%%%%%%%%%%%%%%%%%%%%%%%%
\subsubsection{Buildup}

\conversation{
    %
    \convBackground{(At the beginning of class in the morning)}
    %
    \convExplanation{\dtext{學生}{hok6saang1}}{student}
    %
    \cspeaker{\dtext{學生}{hok6saang1}}
    \cbline{\dtext{何}{ho4}}{Ho, surname}
    \cbline{\dtext{生}{saang1}}{Mr.}
    \cbline{\dtext{何 生}{ho4 saang1}}{Mr. Ho}
    \cbline{\dtext{早晨}{zou2san4}}{good morning}
    \cfline{\dtext{何 生 早晨}{ho4 saang1 zou2san4}}{Good morning, Mr. Ho.}
    \convExplanation{\dtext{先生}{sin1saang1}}{teacher}
    %
    \cspeaker{\dtext{先生}{sin1saang1}}
    \cbline{\dtext{李}{lei5}}{Lee, surname}
    \cbline{\dtext{太}{taai2}}{Mrs.}
    \cbline{\dtext{李 太}{lei5 taai2}}{Mrs. Lee}
    \cfline{\dtext{李 太 早晨。}{lei5 taai2 zou2san4}}{Good morning, Mrs. Lee.}
    %
    \cspeaker{\dtext{學生}{hok6saang1}}
    \cbline{\dtext{對唔住}{deoi3m4zyu6}}{excuse me}
    \cbline{\dtext{我}{ngo5}}{I}
    \cbline{\dtext{係}{hai6}}{am, is, are}
    \cbline{\dtext{唔}{m4}}{not}
    \cbline{\dtext{唔係}{m4hai6}}{am not, is not, are not}
    \cbline{\dtext{我 唔係 李 太。}{ngo5 m4hai6 lei5 taai2}}{I'm not Mrs. Lee.}
    \cfline{\dtext{對唔住,我 唔係 李 太。}{deoi3m4zyu6 ngo5 m4hai6 lei5 taai2}}{Excuse me, I'm not Mrs. Lee.}
    %
    \cbline{\dtext{姓}{sing3}}{have the surname}
    \cbline{\dtext{陳}{can4}}{Chan}
    \cfline{\dtext{我 姓 陳。}{ngo5 sing3 can4}}{My name is Chan.}
    %
    \cspeaker{\dtext{先生}{sin1saang1}}
    \cbline{\dtext{小姐}{sui2ze2}}{Miss; unmarried woman}
    \cbline{\dtext{陳 小姐}{can4 siu2ze2}}{Miss Chan}
    \cbline{\dtext{呀}{aa3}}{Oh, Ah, a mild exclamation}
    \cfline{\dtext{呀,對唔住 陳 小姐。}{aa3 deoi3m4zyu6 can4 siu2ze2}}{Oh, excuse me, Miss Chan.}
    %
    \cspeaker{\dtext{學生}{hok6saang1}}
    \cfline{\dtext{唔緊要}{m4gan2jiu3}}{That's all right. \underline{OR} It doesn't matter.}
    %
    \convBackground{(At the end of the day, the students are leaving class.)}
    %
    \cspeaker{\dtext{學生}{hok6saang1}}
    \cfline{\dtext{再見}{zoi3gin3}}{Goodbye.}
    %
    \cspeaker{\dtext{先生}{sin1saang1}}
    \cfline{\dtext{再見}{zoi3gin3}}{Goodbye.}
}

%%%%%%%%%%%%%%%%%%%%%%%%%%%%%%%%%%%%%%%%
\subsubsection{Recapitulation}

\audioTag{A}{7:30}

\conversation{
    %
    \convBackground{(At the beginning of class in the morning)}
    %
    \cspeaker{\dtext{學生}{hok6saang1}}
    \cfline{\dtext{何 生 早晨}{ho4 saang1 zou2san4}}{Good morning, Mr. Ho.}
    %
    \cspeaker{\dtext{先生}{sin1saang1}}
    \cfline{\dtext{李 太 早晨。}{lei5 taai2 zou2san4}}{Good morning, Mrs. Lee.}
    %
    \cspeaker{\dtext{學生}{hok6saang1}}
    \cfline{\dtext{對唔住,我 唔係 李 太。}{deoi3m4zyu6 ngo5 m4hai6 lei5 taai2}}{Excuse me, I'm not Mrs. Lee.}
    %
    \cfline{\dtext{我 姓 陳。}{ngo5 sing3 can4}}{My name is Chan.}
    %
    \cspeaker{\dtext{先生}{sin1saang1}}
    \cfline{\dtext{呀,對唔住 陳 小姐。}{aa3 deoi3m4zyu6 can4 siu2ze2}}{Oh, excuse me, Miss Chan.}
    %
    \cspeaker{\dtext{學生}{hok6saang1}}
    \cfline{\dtext{唔緊要}{m4gan2jiu3}}{That's all right. \underline{OR} It doesn't matter.}
    %
    \convBackground{(At the end of the day, the students are leaving class.)}
    %
    \cspeaker{\dtext{學生}{hok6saang1}}
    \cfline{\dtext{再見}{zoi3gin3}}{Goodbye.}
    %
    \cspeaker{\dtext{先生}{sin1saang1}}
    \cfline{\dtext{再見}{zoi3gin3}}{Goodbye.}
}