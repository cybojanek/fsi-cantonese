\subsection{Classroom Phrases}

\audioTag{A}{0:05}

The instructor will address you in Cantonese from the first day of class. The following are some instructions which you should learn to respond to. Look at your books while the instructor reads the phrases the first time. Then close your books, and the teacher will give the phrases several more times, using gestures to help you understand. Repease the phrases after him, mimicking his movements as well as his voice, to help you absorb the rhythm and meaning.

\classroomPhrases{
	{\dtext{而家 你哋 聽住 我 講。}{ji4gaa1 nei5dei6 teng1zyu6 ngo5 gong2}}{Now you (plu.) listen while I speak. (i.e., listen, but you don't repeat)}
	{\dtext{而家 我 講,你哋 跟住 我 講。}{ji4gaa1 ngo5 gong2 nei5dei6 gan1zyu6 ngo5 gong2}}{Now I'll speak and you repeat after me.}
	{\dtext{x 本書,x 啲書。}{? bun2syu1, ? di1syu1}}{Close the book, close the books.}
	{\dtext{打開 本 書, 打開 啲 書。}{daa2hoi1 bun2 syu1, daa2hoi1 di1 syu1}}{Close the book, close the books.}
	{\dtext{而家 一 個 一 個 講。}{ji4gaa1 jat1 go3 jat1 go3 gong2}}{Now recite one by one.}
	{\dtext{一齊 講。}{jat1cai4 gong2}}{Recite all together.}
	{\dtext{而家 一齊 跟住 我 講。}{ji4gaa1 jat1cai4 gan1zyu6 ngo5 gong2}}{Now all together repeat after me.}
	{\dtext{再 講 一 次。}{zoi3 gong2 jat1 ci3}}{Say it again.}
	{\dtext{唔好 睇 書。}{m4hou2 tai2 syu1}}{Don't look at your book(s).}
}