\subsubsection{Intonation}

A sentence may be said different ways, to stress different points in the sentence and also to express what the speaker feels about what he is saying. To give an English example, the sentence 'So glad you could come', may be said:

\todo{contour graphs}

\begin{tabularx}{\linewidth}{X X X}
So glad you could \underline{come}. &
\begin{tikzpicture}
\draw (0, 0) -- (3, 0);
\draw (3, 0) -- (4, 0.5);
\draw [->] (4, 0.5) -- (4.5, 0);
\end{tikzpicture}
& normal polite \\

\underline{So} glad you could come. &
\begin{tikzpicture}
\draw (0, 0.5) -- (1, 0);
\draw (1, 0) -- (4, 0);
\draw [->] (4, 0) -- (4.5, -0.5);
\end{tikzpicture}
& effusive polite \\

So glad \underline{you} could come. &
\begin{tikzpicture}
\draw (0, 0) -- (1, 0);
\draw (1, 0) -- (2, 0.5);
\draw (2, 0.5) -- (3, 0);
\draw (3, 0) -- (4, 0);
\draw [->] (4, 0) -- (4.5, -0.5);
\end{tikzpicture}
& (even if your wife couldn't make it) cordial \\

So glad \underline{you} could come. &
\begin{tikzpicture}
\draw (0, 0) -- (1, 0);
\draw (1, 0) -- (2, 0.75);
\draw (2, 0.75) -- (3, 0);
\draw (3, 0) -- (4, 0);
\draw [->] (4, 0) -- (4.5, -0.5);
\end{tikzpicture}
& (even if your \underline{wife} couldn't) sarcastic \\

So glad you \underline{could} come. &
\begin{tikzpicture}
\draw (0, 0) -- (2, 0);
\draw (2, 0) -- (3, 0.5);
\draw [->] (3, 0.5) -- (4.5, -0.5);
\end{tikzpicture}
& (after having though you couldn't) cordial \\

They were glad you could come? &
\begin{tikzpicture}
\draw (0, 0) -- (4, 0);
\draw (4, 0) -- (4.5, 0.5);
\draw [->] (4.5, 0.5) -- (5, 1);
\end{tikzpicture}
& question \\
\end{tabularx}

The graphs of the sentence contours above represent the rise and fall of the voice pitch throughout the length of the sentence. This rise and fall over sentence length we call and "intonation".

You will note that the question sentence (\#5) rises in pitch at the end, and the statement sentences (\#1 - 4) all end with falling pitch, although within their contours rise and fall occurs at different points. In English sentence-final fall is the norm, and sentence-final expresses doubt.

Intonation asl has another job within a sentence, it can express how the speaker feels about what he is saying. By expressive rise and fall of his voice, by varying his "tone of voice", the speaker can indicate that he is angry or happy, doubtful or certain, being polite or rude, suggesting or demanding.

Cantonese sentences too exhbiti intonation contours. Sentence final contours in particular are much more varied in Cantonese than in English, and capable of expressing quite a range of emotional implications.

You may wonder how intonation affects the tone situation in Cantonese, each syllable having as it done its characteristic tone. How the tone contours operate in the framework of sentence contour has been compared to the action of ripples riding on top of waves. Each ripple relates to the one before it and behind it, whether in the trough of the wave or on the crest.

%%%%%%%%%%%%%%%%%%%%%%%%%%%%%%%%%%%%%%%%%%%%%%%%%%%%%%%%%%%%%%%%%%%%%%%%%%%%%%%%
\paragraph{Sentence Stress}

In speaking of sentence stress we mean relative prominence of syllables in a sentence - loud or soft (heavy or light), rapid or slow. Consider the stress pattern of the following English sentences.

\begin{tabularx}{\linewidth}{X X X}
\underline{I'm} John Smith. & (In response to "Which one of you is John Smith?") \\
I'm \underline{John} Smith. & (In response to "I was suppsed to give this letter to Tom Smith") \\
\end{tabularx}

In the sentence above the stressed syllables (those underlined) give prominence to the information requested in the stimulus sentences.

In certain sentences stress differences alone indicate dfference in message content. The pair of sentences often quoted in illustration of this is:

\begin{tabularx}{\linewidth}{X X X}
\underline{Ship} sails today. && (The ship will sail today.) \\
Ship \underline{sails} today. && (Please ship the sails today.) \\
\end{tabularx}

Another example, from a headline in a newspaper:
\indent Boy Scratching Cat is Caught, Destroyed
How do you stress that one?

%%%%%%%%%%%%%%%%%%%%%%%%%%%%%%%%%%%%%%%%%%%%%%%%%%%%%%%%%%%%%%%%%%%%%%%%%%%%%%%%
\paragraph{Sentence Pause}

Another feature important in establishing natural sentence rhythm is pause - the small silences between groups of syllables. Note the following English sentence:

\begin{quote}
	In considering him for the job he took into account his education, previous experience, and appraised potential.
\end{quote}

There is a pause between "job" and "he" in the sentence above, and if you read it instead pausing after "took", you find the sentence doesn't make sense - you have to go back and read it again putting a pause in the right place.

We will not discuss Cantonese stress and pause features in this Introduction, other than to say that Cantonese sentences, like English ones, do exhibit stress and pause phenomena, as well as intonational ones. What this effectively means for you as a student is that you must not concentrate solely on learning words as individual isolated units, but in imitating the teacher's spoken model, you should be alert to his delivery of prhase-length segments and whole sentences, and should mimic the stress, pause, and intonation of the phrases you repeat.
