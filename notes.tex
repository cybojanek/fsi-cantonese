\section{License}

This work is licensed under a Creative Commons Attribution-ShareAlike 4.0 International License. You can find the full text \href{https://creativecommons.org/licenses/by-sa/4.0/}{here}.

In brief summary, you are free to:

\begin{itemize}
	\item Share - copy and redistribute the material in any medium or format
	\item Adapt - remix, transform, and build upon the material for any purpose, even commercially.
\end{itemize}

Under the following terms:

\begin{itemize}
	\item Attribution — You must give appropriate credit, provide a link to the license, and indicate if changes were made. You may do so in any reasonable manner, but not in any way that suggests the licensor endorses you or your use.
	\item ShareAlike — If you remix, transform, or build upon the material, you must distribute your contributions under the same license as the original.
	\item No additional restrictions — You may not apply legal terms or technological measures that legally restrict others from doing anything the license permits.
\end{itemize}

\section{Translator notes}

\begin{itemize}
    \item The original FSI course, along with the accompanying audio is available \href{https://fsi-languages.yojik.eu/languages/FSI/fsi-cantonese.html}{here}.
    \item The original FSI cantonese uses yale romanization with 7 tones. This translation uses jyupting with 6 tones.
    \item This is a work in progress.
    \item Thanks to rathy, kobo-dashi, cato, and anyone else who contributed to the conversation transcripts on the cantonese dictionary \href{http://www.cantonese.sheik.co.uk/phorum/read.php?1,53361,page=1}{forums} back in 2006.
    \item Thanks to Archive.org for having an OCR \href{https://archive.org/stream/Fsi-CantoneseBasicCourse-StudentText/Fsi-CantoneseBasicCourse-Volume1-StudentText_djvu.txt}{version} of the text.
    \item Misc digitization notes are marked as \tnote{Something is different}
    \item Things left to be copied are marked as \todo{Fix this section}
\end{itemize}