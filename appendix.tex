\section{Appendix}

\subsection{Lesson 2}
\label{sec:appendix-lesson2}

\paragraph{1. } At a party:

\listeningConversation{
    \lcEntry {Man} {\dtext{小姐 貴姓 呀?}{siu2ze2 gwai3sing3 aa3}}
    \lcEntry {Woman} {\dtext{我 姓 陳。先生 貴姓 呀?}{ngo5 sing3 can4 sin1saang1 gwai3sing3 aa3}}
    \lcEntry {Man} {\dtext{小姓 何。陳 小姐 係 唔係 廣東人 呀?}{siu2sing3 ho4 can4 siu2ze2 hai6 m4hai6 gwong2dung1jan4 aa3}}
    \lcEntry {Woman} {\dtext{唔係 呀。我 係 上海人。你 呢?你 係 唔係 美國人 呀?}{m4hai6 aa3 ngo5 hai6 soeng6hoi2jan4 nei5 ne1 nei5 hai6 m4hai6 mei5gwok3jan4 aa3}}
    \lcEntry {Man} {\dtext{係 呀。我 係 美國人。}{hai6 aa3 ngo5 hai6 mei5gwok3jan4}}
}

\paragraph{2. } At the first day of school, students are getting acquanited:

\listeningConversation{
    \lcEntry {First student} {\dtext{你 姓 乜嘢 呀?}{nei5 sing3 me1je5 aa3}}
    \lcEntry {Second student} {\dtext{我 姓 王。}{ngo5 sing3 wong4}}
    \lcEntry {First student} {\dtext{你 朋友 呢?}{nei5 pang4jau5 ne1}}
    \lcEntry {Second student} {\dtext{佢 都 係 姓 王 嘅。}{keoi5 dou1 hai6 sing3 wong4 ge3}}
    \lcEntry {First student} {\dtext{你 係 唔係 廣東人 呀?}{nei5 hai6 m4hai6 gwong2dung1jan4 aa3}}
    \lcEntry {Second student} {\dtext{係。}{hai6}}
    \lcEntry {First student} {\dtext{你 朋友 係 唔係 都 係 廣東人 呀?}{nei5 pang4jau5 hai6 m4hai6 dou1 hai6 gwong2dung1jan4 aa3}}
    \lcEntry {Second student} {\dtext{唔係 呀。佢 係 上海人。}{m4hai6 aa3 keoi5 hai6 soeng6hoi2jan4}}
}